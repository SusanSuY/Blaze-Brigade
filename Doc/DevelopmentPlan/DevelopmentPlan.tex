\documentclass{article}

\usepackage[dvipsnames]{xcolor}
\usepackage{booktabs}
\usepackage{graphicx}
\usepackage{tabularx}
\newcolumntype{L}{>{\centering\arraybackslash}m{3cm}}

\newcommand{\bigcell}[2]{\begin{tabular}{@{}#1@{}}#2\end{tabular}}

% ================ Title ===============

\title{ \vspace{40mm}
	\textbf {
	\Huge {\color[rgb]{0.9,0,0}Blaze} Brigade \\
	\large - Development Plan -}}

\date{\today}

\author{SFWR ENG 3XA3 - Section L02 \\
	007 (Group 7) \\ \\
	Jeremy Klotz - klotzjj \\
	Asad Mansoor - mansoa2 \\
	Thien Trandinh - trandit \\
	Susan Yuen - yuens2}

% =============== Document ===============

\begin{document}

\newpage
\maketitle
\pagenumbering{gobble}
\newpage

\begin{table}[hp]
    \begin{tabularx}{\textwidth}{llX}
        \toprule
        \textbf{Date} & \textbf{Developer(s)} & \textbf{Change}\\
        \midrule
        Sept 28, 2016 & All & Created the Development Plan - Rev 0 \\
        Oct 5, 2016 & Susan Yuen & Edited Development Plan to reflect decision to use Unreal Engine 4. \\
        Oct 21, 2016 & Susan Yuen & Edited Development Plan to reflect decision to use XNA Game Studio and Visual Studio for development. \\
        Dec 6, 2016 & Asad Mansoor & Modified Team Meeting Plan to include Location, Deliverables and a detailed Agenda. Included a new risk in the Proof of Concept section. Added content to the Project Review section - Rev 1  \\
        \bottomrule
    \end{tabularx}
    \caption{Revision History} \label{TblRevisionHistory}
\end{table}

\newpage
\pagenumbering{arabic}
\section{Team Meeting Plan}
\begin{table}[h!]
\caption{Team meeting plan with their corresponding time and location.} \label{TblTeamMeetingPlan}
	\centering
	\begin{tabular}{|c|c|c|p{4cm}|}
		\hline
		Meeting ID & Week & Meeting Dates & \multicolumn{1}{|c|}{Location} \\
		\hline
		M1 & Sept 12 - Sept 18 & \bigcell{c}{Sept 14: 10:30am-12:30pm \\ Sept 15: 6:30pm-7:30pm \\ Sept 16: 8:30am-10:30am} & ITB-236 Lab, Online \\
		\hline
		M2 & Sept 19 - Sept 25 & \bigcell{c}{Sept 21: 10:30am-12:30pm \\ Sept 22: 6:30pm-7:30pm \\ Sept 23: 8:30am-10:30am} & ITB-236 Lab, Online \\
		\hline
		M3 & Sept 26 - Oct 2 & \bigcell{c}{Sept 28: 10:30am-12:30pm \\ Sept 29: 6:30pm-7:30pm \\ Sept 30: 8:30am-10:30am} & ITB-236 Lab, Online  \\
		\hline
		M4 & Oct 3 - Oct 9 &\bigcell{c}{Oct 5: 10:30am-12:30pm \\ Oct 6: 6:30pm-7:30pm \\ Oct 7: 8:30am-10:30am} & ITB-236 Lab, Online  \\
		\hline
		M5 & Oct 10 - Oct 16 & \bigcell{c}{Oct 12: 10:30am-12:30pm \\ Oct 13: 6:30pm-7:30pm \\ Oct 14: 8:30am-10:30am} & ITB-236 Lab, Online \\
		\hline
		M6 & Oct 17 - Oct 23 & \bigcell{c}{Oct 19: 10:30am-12:30pm \\ Oct 20: 6:30pm-7:30pm \\ Oct 21: 8:30am-10:30am} & ITB-236 Lab, Online \\
		\hline
		M7 & Oct 24 - Oct 30 & \bigcell{c}{Oct 26: 10:30am-12:30pm \\ Oct 27: 6:30pm-7:30pm \\ Oct 28: 8:30am-10:30am} & ITB-236 Lab, Online \\
		\hline
		M8 & Oct 31 - Nov 6 & \bigcell{c}{Nov 2: 10:30am-12:30pm \\ Nov 3: 6:30pm-7:30pm \\ Nov 4: 8:30am-10:30am} & ITB-236 Lab, Online \\
		\hline
		M9 & Nov 7 - Nov 13 & \bigcell{c}{Nov 9: 10:30am-12:30pm \\ Nov 10: 6:30pm-7:30pm \\ Nov 11: 8:30am-10:30am} & ITB-236 Lab, Online \\
		\hline
		M10 & Nov 14 - Nov 20 & \bigcell{c}{Nov 16: 10:30am-12:30pm \\ Nov 17: 6:30pm-7:30pm \\ Nov 18: 8:30am-10:30am} & ITB-236 Lab, Online \\
		\hline
		M11 & Nov 21 - Nov 27 & \bigcell{c}{Nov 23: 10:30am-12:30pm \\ Nov 24: 6:30pm-7:30pm \\  Nov 25: 8:30am-10:30am} & ITB-236 Lab, Online \\
		\hline
		M12 & Nov 28 - Dec 4 & \bigcell{c}{Nov 30: 10:30am-12:30pm \\ Dec 1: 6:30pm-7:30pm \\ Dec 2: 8:30am-10:30am} & ITB-236 Lab, Online \\
		\hline
		M13 & Dec 4 - Dec 7 & \bigcell{c}{Dec 6: 6:30pm-7:30pm \\ Dec 7: 10:30am-12:30am} & ITB-236 Lab, Online \\
		\hline
	\end{tabular}
\end{table}

\begin{table}
\caption{Team meeting agenda with specified roles and deliverables} \label{TblTeamMeetingAgenda}
\begin{tabular}{|L|L|L|L|}
		\hline
		Meeting ID & Meeting Agenda & Roles & Deliverables \\
		\hline
		M1 & Form the Blaze Brigade team and introduce team members & All - Engage in the formation and discussion & N/A\\
		\hline
		M2 & Finalize on the project and develop problem statement  & All - Engage in the discussion and share ideas for solution & Problem Statement\\
		\hline
		M3 & Create development plan and finalize on programming methodologies & All - Assign issues and discuss plan & Development Plan \\
		\hline
		M4 & Create Software Requirements Specification & All - Assign issues and discuss scope of requirements & Requirements Rev 0 \\
		\hline
		M5 & Implement Proof of Concept. & All - Download tools and code specific parts & N/A \\
		\hline
		M6 & Structure Proof of Concept into a presentation & All - Prepare to demo and answer potential questions & Proof of Concepts Demo\\
		\hline
		M7 & Discuss test methodologies and test cases & All - Assign issues and discuss scope of tests & Test Plan Rev 0\\
		\hline
		M8 & Discuss any issues within system. Plan to integrate test cases & All - Share feedback of current build & N/A \\
		\hline
		M9 & Discuss design pattern and decompose system by creating MIS and MG & All - Assign MIS and MG to various members & Design Document Rev 0\\
		\hline
		M10 & Structure Rev 0 into a presentation & All - Prepare to demo and answer potential questions & Rev 0 Demonstration \\
		\hline
		M11 & Discuss any issues within system. Plan to integrate changes & All - Provide feedback from Rev 0 & N/A \\
		\hline
		M12 & Formulate and present the final demonstration of Blaze Brigade & All - Develop and present portion of presentation & Final Demonstration Rev 1 \\
		\hline
		M13 & Create Test Report and make changes to Rev 0 documents & All - Assign to review documentation & Final Documentation Rev 1 \\
		\hline
\end{tabular}
\end{table}

\newpage

	
\section{Team Communication Plan}
The team will use Skype for communication when outside of the agreed upon meeting times. These lines of communication are available when team members require assistance with their assigned work or require input from other team members on a topic of question. The team will also be using Slack to organize announcements regarding project development and deliverable deadlines.
	
\section{Team Member Roles}
\begin{itemize}
    \item \textbf{Jeremy Klotz}: Algorithms Specialist, Developer
    \item \textbf{Asad Mansoor}: Tester, Developer
    \item \textbf{Thien Trandinh}: Gameplay Mechanic, Developer
    \item \textbf{Susan Yuen}: Git Master, Product Architect, Developer
\end{itemize}
	
\section{Git Workflow Plan}
After considering the different types of workflows, we concluded that \textbf{centralized workflow} best fits the requirements of this project. This is due to the fact that the project is relatively small - only spanning 12 weeks, and the team will see the project through from start to finish over this time period. As such, a release branch  separate from a development branch is not necessary. In addition, team members will be working on aspects of the game pertaining to the same feature or features that rely on each other, so creating any additional feature branches are also unnecessary. Due to these reasons, feature-branch and gitflow are excluded. As a result, we decided on maintaining only one branch, and are thus implementing the centralized workflow for our project. Labels will be used to label any commits containing documents that are graded.
	
\section{Proof of Concept Demonstration Plan}
The proof of concept demonstration shall consist of the layout of the software architecture, including the skeleton of the majority of classes, functions, and implementation of the Model-View-Controller software design. The program shall have a working grid implemented, as well click detection and mouse functionality. The game shall also have one unit, which the player shall be able to move by first selecting the unit, then selecting another position on the grid to move the unit to. The player shall also be able to select and deselect the unit by clicking on the unit repeatedly.

\subsection*{Will a part of the implementation be difficult?}
There is no significant risk other than implementing all minor details within the given time constraint. An unforeseen minor risk can be losing the license to the free software such as Visual Studio and XNA Game Studio provided by the McMaster Engineering Faculty. If that does occur, a new copy may need to be purchased or there could be a change of the technologies being used to deliver the solution. \\
    
\subsection*{Will testing be difficult?}
Testing will not be difficult as the team members have experience with unit testing in frameworks such as JUnit. Although our project will be coded in C\#, the unit tests will share similar concepts and ideas to our previous experience with JUnit. As such, our automated unit tests will be able to cover whitebox testing of single functions and state variables. \\
    
\subsection*{Is a required library difficult to install?}
The programs that will be used to develop, which include Visual Studio 2015 and XNA Game Studio, are available for free download to students in McMaster's Software and Computing Department, and therefore not be difficult. The downloads are available online through McMaster's CAS department. \\

\subsection*{Will portability be a concern?}
Portability will be a concern as the platform that the game supports only includes Windows, due to the restriction of the software the game is developed with. Visual Studio and XNA Game Studio are supported by Microsoft, therefore the game will only be able to support Windows OS.
    
\section{Technology}
\begin{itemize}
    \item \textbf{Programming Language}: C\#
    \item \textbf{IDE}: Visual Studio 2015, XNA Game Studio
    \item \textbf{Testing Framework}: Visual Studio Unit Testing Framework
    \item \textbf{Documentation}: LaTeX
    \item \textbf{Other}: Git, Photoshop
\end{itemize}
	
\section{Coding Style}
The coding style for the project will follow Microsoft's C\# Programming Guide (https://msdn.microsoft.com/en-us/library/ff926074.aspx).
	
\section{Project Schedule}
Please refer to the .gan file for access to the project Gantt Chart within the DevelopmentPlan folder.
	
\section{Project Review}
Upon reaching Revision 1 of the Blaze Brigade project, a few changes have been made to various documentation either as an update to outdated information or implementation on the feedback provided. Considering its timeframe, the project has been a success as the team has completed all of the objectives and goals that were initially planned. Although this was a first time for many team members being involved in a large-scale software solution, the feedback and personal experience throughout this learning process has shown what has worked well, and what did not work out as well will be improved on for future projects. \\

\noindent
The consistency in delivering the right amount of content of each documentation has been a main contributor to the success of the project. Normally, all tasks would be assigned with the deadline set earlier than the actual due date to provide a buffer for edits or updates. The correct use of technology has worked well in the implementation of the project and provided no limitation in delivering the assigned solution. Each team member's role were assigned and everyone was well aware on where they stand in delivering their assigned contribution. The team would continue using the concepts presented within this development plan as a reference to future projects or even Blaze Brigade if it were to continue to Revision 2. \\

\noindent
Areas of improvements can be seen in the planning of team communication and git workflow. Team communication was active during the labs, but the times during the evenings or weekend would be minimal. This can be changed if everyone were to agree to have a companion communication application on their desktop as well as their mobile devices. Moreover, a scope of switching communication application is also a possibility to combine the chatting aspect with the organization of tasks. Learning how to use git was a learning curve for some team members and prompted for a better guide during the initial planning phase of the project. There also has to be a method such to check each person's commits and level of contribution to ensure the right metric on what gets achieved for each issue. \\

\noindent
In regards to a future project, the team would better modify the development plan to be a little more detailed and updated in terms of its correctness, as the development plan was essential on how the team should work. The meeting plan should contain more related agenda information, to ensure all the following points get covered in the meeting. In addition, there is a scope of incorporating meeting minutes into future meetings, such that everyone is well aware of the decisions taken place. \\

\noindent
Overall, the quality of the team meetings have been productive and clear. Changes within the team communication and time management should be prioritized in the next revision to meet the team's full potential. One particular change is switch the central communication application to include both the chatting aspect as well as organizing tasks. This could mean to fully transfer communications onto Slack or Basecamp. Furthermore, a larger scope should be placed on how tasks should get assigned to avoid unaligned balance of tasks given to a specific team member.

\end{document}